% Template:     Poster LaTeX
% Documento:    Archivo de ejemplo
% Versión:      0.1.4 (31/10/2021)
% Codificación: UTF-8
%
% Autor: Pablo Pizarro R.
%        pablo@ppizarror.com
%
% Manual template: [https://latex.ppizarror.com/poster]
% Licencia MIT:    [https://opensource.org/licenses/MIT]

\begin{block}{Introducción}
	Este template permite crear pósters usando las funciones y entornos de toda la suite \href{https://latex.ppizarror.com/}{Template-Latex}. Los entornos para insertar imágenes, códigos fuente, tablas y otros elementos son exactamente los mismos, lo que facilita el uso y la migración entre estilos.
	
	\insertimage{test-image}{width=0.75\colwidth}{Ejemplo de imagen.}
\end{block}

\begin{exampleblock}{}
	Template-Poster, basado en el proyecto \href{https://github.com/anishathalye/gemini}{gemini}, ofrece la posibilidad de ser compilado en múltiples sistemas, como overleaf o texstudio sin requerir de complicados comandos por terminal ni de makefiles.
	
	\begin{itemize}
		\item Fácil de usar \cite{template}
		\item Múltiples comandos adicionales
		\item Totalmente personalizable
		\item Soporte instantáneo a bibliografía \cite{einstein}
	\end{itemize}
\end{exampleblock}

\begin{block}{Insertar tablas}
	Insertar tablas es muy fácil, para ello se deben usar exactamente los mismos comandos que en todo el resto de templates, dando soporte a celdas de colores, notas al pie, entre otros. También funcionan los plugins \cite{tablesgenerator, excel2latex}.
	
	% Tabla generada con el plugin Excel2Latex
	\begin{table}[H]
		\centering
		\caption{Ejemplo de tablas.}
		\begin{tabular}{ccc}
			\hline
			\textbf{Columna 1} & \textbf{Columna 2} & \textbf{Columna 3} \bigstrut\\
			\hline
			$\omega$ & $\nu$ & $\delta$ \bigstrut[t]\\
			$\Phi$ & $\Theta$ & $\varSigma$ \\
			$\xi$ & $\kappa$ & $\varpi$ \bigstrut[b] \\
			\hline
		\end{tabular}
		\label{tab:tabla-1}
	\end{table}

	O bien, con notas:
	
	\begin{table}
		\begin{threeparttable}
			\centering
			\caption{Ejemplo de tabla que usa múltiples columnas y tiene notas.}
			\begin{tabular}{cccc}
				\hline
				\textbf{Columna 1} & \textbf{Columna 2} & \textbf{Columna 3} & \textbf{Columna 4} \bigstrut\\
				\hline
				1 & $\omega$ & $\nu$ & $\delta$ \tnote{a} \\
				2 & $\Phi$ & $\Theta$ & $\varSigma$ \tnote{1} \\
				3 & $\xi$ & $\kappa$ & $\varpi$ \\
				\hline
			\end{tabular}
			\begin{tablenotes}
				Esta tabla acepta comentarios y notas al margen.
				\item[a] Este elemento tiene una descripción debajo de la tabla
				\item[1] Más comentarios
			\end{tablenotes}
		\end{threeparttable}
	\end{table}

	Para insertar ecuaciones, se puede de la forma tradicional o bien usando cualquier función del template \ref{eqn:identidad-imposible}:
	
	\insertequation[\label{eqn:identidad-imposible}]{\pow{a}{k}=\pow{b}{k}+\pow{c}{k} \quad \forall k>2}
	\insertequationcaptioned[\label{eqn:formulasinsentido}]{\int_{a}^{b} f(x) \dd{x} = \fracnpartial{f(x)}{x}{\eta} \cdotp \textstyle \sum_{x=a}^{b} f(x)\cancelto{1+\frac{\epsilon}{k}}{\bigp{1+\Delta x}}}{Ecuación sin sentido \cite{einstein}.}
\end{block}


% CREA UNA NUEVA COLUMNA
\separatorcolumn


\begin{block}{Párrafos e imágenes múltiples}
	El entorno \textbf{images} ofrece funciones para insertar imágenes múltiples:
	
	\begin{images}{Ejemplo de imagen múltiple.}
		\addimage{test-image}{height=7cm}{Ciudad}
		\addimageanum{test-image-wrap}{height=7cm}
	\end{images}

	Para párrafos, basta usar \textbf{multicol}:
	
	\begin{multicols}{2}
		\lipsum[1]
		
		\begin{enumeratebf}[label=\alph*) ] % Fuente en negrita
			\item Peras
			\item Manzanas
			\item Naranjas
		\end{enumeratebf}
		
		\begin{enumerate}[label=\roman*) ]
			\item Rojo
			\item Café
			\item Morado
		\end{enumerate}
		
		\begin{enumerate}[label=\greek*) ]
			\item Matemáticas
			\item Lenguaje
			\item Filosofía
		\end{enumerate}
	\end{multicols}

	Para insertar códigos fuente el procedimiento es igual que el resto de templates:

\begin{sourcecode}{matlab}{Ejemplo en Matlab.}
% Se crea gráfico
f = figure(1); hold on; movegui(f, 'center');
xlabel('td/Tn'); ylabel('FAD=Umax/Uf0');

for j = 1:length(BETA)
	fad = ones(1, NDATOS); % Arreglo para el FAD
	for i = 1:NDATOS
		[t, u_t, ~, ~] = main(BETA(j), r(i), M, K, F0, 0);
		fad(i) = max(abs(u_t)) / uf0;
	end
end
\end{sourcecode}

	\lipsum[11]
\end{block}

\begin{block}{Referencias}
	\bibliography{library}
\end{block}