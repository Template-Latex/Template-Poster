% Template:     Poster LaTeX
% Documento:    Funciones exclusivas de Template-Poster
% Versión:      0.0.5 (27/09/2021)
% Codificación: UTF-8
%
% Autor: Pablo Pizarro R.
%        pablo@ppizarror.com
%
% Manual template: [https://latex.ppizarror.com/poster]
% Licencia MIT:    [https://opensource.org/licenses/MIT]

% Configura el poster
\title{\documenttitle}
\setbeamercolor{background canvas}{bg=\pagescolor}

% Entorno del poster
\newenvironment{poster}[1][]{%
	\begin{frame}[fragile,t,environment=poster,#1]%
	\begin{columns}[t]%
	\begin{column}{\sepwidth}\end{column}%
	\begin{column}{\colwidth}%
	}{
	\end{column}
	\begin{column}{\sepwidth}\end{column}%
	\end{columns}%Ç
	\end{frame}%
}

% Separador de columnas
\newcommand{\separatorcolumn}{%
	\end{column}%
	\begin{column}{\sepwidth}\end{column}%
	\begin{column}{\colwidth}%
}

% Importa librerías
\usepackage[size=custom,width=\posterwidth,height=\posterheight,scale=1.0]{beamerposter}
\usepackage{exscale}
\changefontsizes{\documentfontsize pt}

% Macros generales
\newcommand{\samelineand}{%
	\qquad%
}

% Heading
%	#1	Contenido
\newcommand{\heading}[1]{%
	\par\bigskip%
	{\usebeamerfont{heading}\usebeamercolor[fg]{heading}#1}%
	\par%
	\smallskip%
}

% Configura los logos
\newlength{\logoleftwidth}
\setlength{\logoleftwidth}{0cm}
\newlength{\logorightwidth}
\setlength{\logorightwidth}{0cm}
\newlength{\maxlogowidth}
\setlength{\maxlogowidth}{0cm}

% Define logo izquierdo
%	#1	Parámetros de la imagen
%	#2	Archivo de imagen
\newcommand{\logoleft}[2]{%
	\newcommand{\insertlogoleft}{\includegraphics[#1]{#2}}%
	\settowidth{\logoleftwidth}{\insertlogoleft}%
	\addtolength{\logoleftwidth}{10ex}%
	\setlength{\maxlogowidth}{\maxof{\logoleftwidth}{\logorightwidth}}%
}

% Define logo derecho
%	#1	Parámetros de la imagen
%	#2	Archivo de imagen
\newcommand{\logoright}[2]{%
	\newcommand{\insertlogoright}{\includegraphics[#1]{#2}}%
	\settowidth{\logorightwidth}{\insertlogoright}%
	\addtolength{\logorightwidth}{10ex}%
	\setlength{\maxlogowidth}{\maxof{\logoleftwidth}{\logorightwidth}}%
}

% Define el headline
\setbeamertemplate{headline}{%
\begin{beamercolorbox}{headline}%
	\begin{columns}%
		\begin{column}{\maxlogowidth}%
			\vskip5ex%
			\centering%
			\ifdefined%
				\insertlogoleft%
				\vspace*{\fill}%
				\hspace*{0ex}%
				% \raggedright
				\insertlogoleft%
				\vspace*{\fill}%
			\else\fi%
		\end{column}%
		\begin{column}{\dimexpr\paperwidth-\maxlogowidth-\maxlogowidth}%
			\usebeamerfont{headline}%
			\vskip3ex%
			\centering%
			{\usebeamerfont{headline title}\usebeamercolor[fg]{headline title}%
				\inserttitle\\[0.5ex]%
			}%
			\vspace{1cm}%
			{\usebeamerfont{headline author}\usebeamercolor[fg]{headline author}%
				\insertauthor\\[1ex]%
			}%
			\vspace{0.1cm}%
			{\usebeamerfont{headline institute}\usebeamercolor[fg]{headline institute}%
				\insertinstitute\\[1ex]%
			}%
		\end{column}
		\begin{column}{\maxlogowidth}%
			\centering%
			\vskip5ex%
			\ifdefined\insertlogoright%
				\vspace*{\fill}%
				% \raggedleft%
				\insertlogoright%
				\hspace{10ex}%
				\vspace*{\fill}%
			\else\fi%
		\end{column}
	\end{columns}
	\vspace*{3.5ex}%
	\ifbeamercolorempty[bg]{headline rule}{}{%
		\begin{beamercolorbox}[wd=\paperwidth,colsep=0.5ex]{headline rule}\end{beamercolorbox}%
	}%
\end{beamercolorbox}%
}

% Bloques
\setbeamertemplate{block begin}{%
	\begin{beamercolorbox}[colsep*=0ex,dp=2ex,center]{block title}%
		\vskip 0pt%
		\ifthenelse{\equal{\insertblocktitle}{}}{}{%
			\usebeamerfont*{block title}\insertblocktitle%
			\vskip -1.25ex%
			\begin{beamercolorbox}[colsep=0.025ex]{block separator}\end{beamercolorbox}%
		}
	\end{beamercolorbox}%
	{\parskip 0pt \par}%
	\ifthenelse{\equal{\insertblocktitle}{}}{%
		\vspace{-1.65\baselineskip}%
	}{}
	\usebeamerfont{block body}%
	\vskip -0.5ex%
	\begin{beamercolorbox}[colsep*=0ex]{block body}%
		\justifying%
		\setlength{\parskip}{1ex}%
		\vskip -2ex%
	}%
	\setbeamertemplate{block end}%
	{%
	\end{beamercolorbox}%
	\vskip 0pt%
	\vspace*{2ex}%
}

% Bloque Alerta
\setbeamertemplate{block alerted begin}{%
	\begin{beamercolorbox}[colsep*=0ex,dp=2ex,center]{block alerted title}%
		\vskip 0pt%
		\ifthenelse{\equal{\insertblocktitle}{}}{}{%
			\usebeamerfont{block title}\insertblocktitle%
			\vskip -1.25ex%
			\begin{beamercolorbox}[colsep=0.025ex]{block alerted separator}\end{beamercolorbox}%
			\vskip -0.75ex%
		}
	\end{beamercolorbox}%
	{\parskip 0pt \par}%
	\ifthenelse{\equal{\insertblocktitle}{}}{%
		\vspace{-1.65\baselineskip}%
	}{}
	\usebeamerfont{block body}%
	\vskip -0.5ex%
	\begin{beamercolorbox}[colsep*=0ex]{block alerted body}%
		\justifying%
		\begin{adjustwidth}{1ex}{1ex}%
			\setlength{\parskip}{1ex}%
			\vskip -0.35ex%
		}%
		\setbeamertemplate{block alerted end}%
		{%
		\end{adjustwidth}%
		\vskip 1.25ex%
	\end{beamercolorbox}%
	% \vskip 0pt%
	\vspace*{2ex}%
}

% Bloque Ejemplo
\setbeamertemplate{block example begin}{%
	\begin{beamercolorbox}[colsep*=0ex,dp=2ex,center]{block example title}%
		\vskip 0pt%
		\ifthenelse{\equal{\insertblocktitle}{}}{}{%
			\usebeamerfont{block title}\insertblocktitle%
			\vskip -1.25ex%
			\begin{beamercolorbox}[colsep=0.025ex]{block example separator}\end{beamercolorbox}%
			\vskip -0.75ex%
		}
	\end{beamercolorbox}%
	{\parskip 0pt \par}%
	\ifthenelse{\equal{\insertblocktitle}{}}{%
		\vspace{-1.65\baselineskip}%
	}{}
	\usebeamerfont{block body}%
	\vskip -0.5ex%
	\begin{beamercolorbox}[colsep*=0ex]{block example body}%
		\justifying%
		\begin{adjustwidth}{1ex}{1ex}%
			\setlength{\parskip}{1ex}%
			\vskip -0.35ex%
		}%
		\setbeamertemplate{block example end}%
		{%
		\end{adjustwidth}%
		\vskip 1.25ex%
	\end{beamercolorbox}%
	% \vskip 0pt%
	\vspace*{2ex}%
}

% Footer
%	#1	Contenido del footer
\newcommand{\footercontent}[1]{%
	\newcommand{\insertfootercontent}{#1}%
}

\setbeamertemplate{footline}{%
	\ifdefined\insertfootercontent%
	\begin{beamercolorbox}[vmode]{headline}%
		\ifbeamercolorempty[bg]{headline rule}{}{%
			\begin{beamercolorbox}[wd=\paperwidth,colsep=0.25ex]{headline rule}\end{beamercolorbox}%
		}%
		\vspace{1.5ex} % 0.5ex
		\hspace{\sepwidth}%
		\usebeamerfont{footline}%
		\centering%
		\insertfootercontent%
		\hspace{\sepwidth}%
		\vspace{1.5ex} %0.75ex
	\end{beamercolorbox}%
	\else\fi%
}