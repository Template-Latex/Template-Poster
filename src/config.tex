% Template:     Poster LaTeX
% Documento:    Configuraciones del template
% Versión:      1.2.6 (07/02/2025)
% Codificación: UTF-8
%
% Autor: Pablo Pizarro R.
%        pablo@ppizarror.com
%
% Manual template: [https://latex.ppizarror.com/poster]
% Licencia MIT:    [https://opensource.org/licenses/MIT]

% CONFIGURACIONES GENERALES
\def\documentfontsize {23}         % Tamaño de la fuente del documento [pt]
\def\documentinterline {1}         % Interlineado del documento [factor]
\def\documentparindent {0}         % Tamaño del indentado de párrafos [pt]
\def\documentparskip {0}           % Tamaño adicional entre párrafos (+/-) [pt]
\def\fontdocument {ralewaylight}   % Tipografía base, ver soportadas en manual
\def\fonttypewriter {tmodern}      % Tipografía de \texttt, ver manual
\def\fonturl {same}                % Tipo de fuente url {tt,sf,rm,same}
\def\frametextjustified {true}     % Justifica todos los párrafos de los frames
\def\graphicxdraft {false}         % En true no carga las imágenes (modo draft)
\def\pointdecimal {true}           % N° decimales con punto en vez de coma
\def\showlayoutlines {false}       % Muestra el layout de la página

% CONFIGURACIÓN DE LAS LEYENDAS - CAPTION
\def\captionalignment {justified}  % Posición {centered,justified,left,right}
\def\captionfontsize{small}        % Tamaño de fuente de los caption
\def\captionlabelformat {simple}   % Formato leyenda {empty,simple,parens}
\def\captionlabelsep {colon}       % Sep. {none,colon,period,space,quad,newline}
\def\captionlessmarginimage {0.1}  % Margen sup/inf de figura sin leyenda [cm]
\def\captionlrmargin {0}           % Márgenes izq/der de la leyenda [cm]
\def\captionlrmarginmc {0}         % Margen izq/der leyenda dentro de cols. [cm]
\def\captionmarginimage {0}        % Margen vertical entre caption e imagen [cm]
\def\captionmarginimages {0}       % Margen vertical entre caption e images [cm]
\def\captionmarginimagesmc {0}     % Margen vert. entre caption e imagesmc [cm]
\def\captionmarginmultimg {0}      % Margen izq/der leyendas múltiple img [cm]
\def\captionnumcode {arabic}       % N° código {arabic,alph,Alph,roman,Roman}
\def\captionnumequation {arabic}   % N° ecuac. {arabic,alph,Alph,roman,Roman}
\def\captionnumfigure {arabic}     % N° figuras {arabic,alph,Alph,roman,Roman}
\def\captionnumsubfigure {alph}    % N° subfigs. {arabic,alph,Alph,roman,Roman}
\def\captionnumsubtable {alph}     % N° subtabla {arabic,alph,Alph,roman,Roman}
\def\captionnumtable {arabic}      % N° tabla {arabic,alph,Alph,roman,Roman}
\def\captionsubchar {.}            % Carácter entre N° objeto - subfigura/tabla
\def\captiontbmarginfigure {20}    % Margen sup/inf de leyenda en figuras [pt]
\def\captiontbmargintable {7}      % Margen sup/inf de la leyenda en tablas [pt]
\def\captiontextbold {false}       % Etiqueta (código,figura,tabla) en negrita
\def\captiontextsubnumbold {false} % N° subfigura/subtabla en negrita
\def\codecaptiontop {true}         % Leyenda arriba del código fuente
\def\equationcaptioncenter {true}  % Ecuaciones están centradas o justificadas
\def\figurecaptiontop {false}      % Leyenda arriba de las imágenes
\def\marginaligncaptbottom {0.1}   % Margen inferior caption en align [cm]
\def\marginaligncapttop {-0.75}    % Margen superior caption en align [cm]
\def\marginalignedcaptbottom {0.1} % Margen inferior caption en aligned [cm]
\def\marginalignedcapttop {-0.75}  % Margen superior caption en aligned [cm]
\def\margineqncaptionbottom {0}    % Margen inferior caption ecuación [cm]
\def\margineqncaptiontop {-0.7}    % Margen superior caption ecuación [cm]
\def\margingathercaptbottom {0.1}  % Margen inferior caption en gather [cm]
\def\margingathercapttop {-0.9}    % Margen superior caption en gather [cm]
\def\margingatheredcaptbottom{0.1} % Margen inferior caption en gathered [cm]
\def\margingatheredcapttop {-0.7}  % Margen superior caption en gathered [cm]
\def\sectioncaptiondelimiter {.}   % Carácter delimitador n° objeto y sección
\def\showsectioncaptioncode {none} % N° sec código {none,chap,(s/ss/sss/ssss)ec}
\def\showsectioncaptioneqn {none}  % N° sec ecuac. {none,chap,(s/ss/sss/ssss)ec}
\def\showsectioncaptionfig {none}  % N° sec figs. {none,chap,(s/ss/sss/ssss)ec}
\def\showsectioncaptionmat {none}  % N° matemático {none,chap,(s/ss/sss/ssss)ec}
\def\showsectioncaptiontab {none}  % N° sec tablas {none,chap,(s/ss/sss/ssss)ec}
\def\subcaptionfsize {scriptsize}  % Tamaño de la fuente de los subcaption
\def\subcaptionlabelformat{parens} % Formato leyenda sub. {empty,simple,parens}
\def\subcaptionlabelsep {space}    % Sep. {none,colon,period,space,quad,newline}
\def\tablecaptiontop {true}        % Leyenda arriba de las tablas

% ANEXO, CITAS, REFERENCIAS
\def\bibtexrefsep {0}              % Separación entre refs. {bibtex} [pt]
\def\bibtexstyle {ieeetr}          % Formato refs. bibtex {apa,ieeetr,etc...}
\def\stylecitereferences {bibtex}  % Estilo cita/ref {bibtex,custom}

% CONFIGURACIONES DE OBJETOS
\def\animatedimageautoplay {true}  % Autoplay en imágenes animadas
\def\animatedimagecontrols {false} % Muestra los controles en imágenes animadas
\def\animatedimageloop {true}      % Hace loops en imágenes animadas
\def\columnsepwidth {2.1}          % Separación entre columnas [em]
\def\defaultimagefolder {img/}     % Carpeta raíz de las imágenes
\def\equationleftalign {false}     % Ecuaciones alineadas a la izquierda
\def\equationrestart {none}        % Reinicio n° {none,chap,(s/ss/sss/ssss)ec}
\def\footnotelmargin {10}          % Margen entre footnote y el número [pt]
\def\footnoterestart {none}        % N° foot. {none,chap,page,(s/ss/sss/ssss)ec}
\def\footnoterulefigure {false}    % Footnote en figuras tienen línea superior
\def\footnoterulepage {false}      % Footnote en páginas tienen línea superior
\def\footnoteruletable {false}     % Footnote en tablas tienen línea superior
\def\footnotetwocolumn {false}     % Footnote en dos columnas
\def\fpremovetopbottomcenter{true} % Elimina espacio vert. al centrar con b!,t!
\def\imagedefaultplacement {H}     % Posición por defecto de las imágenes
\def\marginalignbottom {-0.4}      % Margen inferior entorno align [cm]
\def\marginalignedbottom {-0.2}    % Margen inferior entorno aligned [cm]
\def\marginalignedtop {-0.4}       % Margen superior entorno aligned [cm]
\def\marginaligntop {-0.4}         % Margen superior entorno align [cm]
\def\marginequationbottom {-0.2}   % Margen inferior ecuaciones [cm]
\def\marginequationtop {0}         % Margen superior ecuaciones [cm]
\def\marginfloatimages {-13}       % Margen sup figs. insertimageleft/right [pt]
\def\margingatherbottom {-0.2}     % Margen inferior entorno gather [cm]
\def\margingatheredbottom {-0.1}   % Margen inferior entorno gathered [cm]
\def\margingatheredtop {-0.4}      % Margen superior entorno gathered [cm]
\def\margingathertop {-0.4}        % Margen superior entorno gather [cm]
\def\marginimagebottom {-0.50}     % Margen inferior figura [cm]
\def\marginimagemultbottom {0}     % Margen inferior imágenes múltiples [cm]
\def\marginimagemultright {1.25}   % Margen derecho imágenes múltiples [cm]
\def\marginimagemulttop {0}        % Margen superior imágenes múltiples [cm]
\def\marginimagetop {0}            % Margen superior figuras [cm]
\def\numberedequation {true}       % Ecuaciones con \insert... numeradas
\def\senumerti {\arabic{enumi}.}   % Estilo enumerate nivel 1
\def\senumertii {\alph{enumii})}   % Estilo enumerate nivel 2
\def\senumertiii{\roman{enumiii}.} % Estilo enumerate nivel 3
\def\senumertiv {\Alph{enumiv})}   % Estilo enumerate nivel 4
\def\sitemizei {\iitembsquare}     % Estilo itemize nivel 1
\def\sitemizeii {\iitembcirc}      % Estilo itemize nivel 2
\def\sitemizeiii {\iitemdash}      % Estilo itemize nivel 3
\def\sitemizeiv {\iitemcirc}       % Estilo itemize nivel 4
\def\sitemsmargini {85}            % Margen ítems nivel 1 [pt]
\def\sitemsmarginii {50.6}         % Margen ítems nivel 2 [pt]
\def\sitemsmarginiii {43}          % Margen ítems nivel 3 [pt]
\def\sitemsmarginiv {0}            % Margen ítems nivel 4 [pt]
\def\sourcecodebgmarginbottom {0}  % Margen inferior del bloque de color [pt]
\def\sourcecodebgmarginleft {-1}   % Margen izquierdo del bloque de color [pt]
\def\sourcecodebgmarginright {0}   % Margen derecho del bloque de color [pt]
\def\sourcecodebgmargintop {0}     % Margen superior del bloque de color [pt]
\def\sourcecodefontf {\ttfamily}   % Tipo de letra código fuente
\def\sourcecodefonts {\normalsize} % Tamaño letra código fuente
\def\sourcecodeilfontf {\ttfamily} % Tipo de letra código fuente inline
\def\sourcecodeilfonts{\normalsize}% Tamaño letra código fuente inline
\def\sourcecodenumbersep {12}      % Separación entre número línea y código [pt]
\def\sourcecodenumbersize{\scriptsize} % Tamaño fuente número línea
\def\sourcecodeskipabove {0.75}    % Espacio sobre recuadro código [em]
\def\sourcecodeskipbelow {0.5}     % Espacio bajo recuadro código [em]
\def\sourcecodetabsize {3}         % Tamaño tabulación código fuente
\def\tabledefaultplacement {H}     % Posición por defecto de las tablas
\def\tablenotesameline {true}      % Notas en tablas en una sola línea
\def\tablenotesfontsize{\scriptsize} % Tamaño de fuente de las notas en tablas
\def\tablepaddingh {0.75}          % Espaciado horizontal de celda de las tablas
\def\tablepaddingv {1.15}          % Espaciado vertical de celda de las tablas
\def\tikzdefaultplacement {H}      % Posición por defecto de las figuras tikz

% CONFIGURACIÓN DE LOS COLORES DEL DOCUMENTO
\def\captioncolor {mitred}         % Color nombre objeto (código,figura,tabla)
\def\captiontextcolor {black}      % Color de la leyenda
\def\enumerateitemcolor {mitred}   % Color de los enumerate por defecto
\def\highlightcolor {yellow}       % Color del subrayado con \hl
\def\itemizeitemcolor {mitred}     % Color de los ítems por defecto
\def\linkcolor {black}             % Color de los links del documento
\def\maintextcolor {black}         % Color principal del texto
\def\numcitecolor {black}          % Color del n° de las referencias o citas
\def\pagescolor {white}            % Color de la página
\def\showborderonlinks {false}     % Color de un link por un recuadro de color
\def\sourcecodebgcolor {lgray}     % Color de fondo del código fuente
\def\tablelinecolor {black}        % Color de las líneas de las tablas
\def\tablerowfirstcolor {none}     % Primer color de celda de las tablas
\def\tablerowsecondcolor {gray!10} % Segundo color de celda de las tablas
\def\urlcolor {magenta}            % Color de los enlaces web (\href,\url)

% OPCIONES DEL PDF COMPILADO
\def\cfgbookmarksopenlevel {1}     % Nivel marcadores en pdf (1:secciones)
\def\cfgpdfbookmarkopen {true}     % Expande marcadores del nivel configurado
\def\cfgpdfcenterwindow {true}     % Centra ventana del lector al abrir el pdf
\def\cfgpdfcopyright {}            % Establece el copyright del documento
\def\cfgpdfdisplaydoctitle {true}  % Muestra título del informe en visor
\def\cfgpdffitwindow {true}        % Ajusta la ventana del lector tamaño pdf
\def\cfgpdfkeywords {}             % Palabras clave del pdf
\def\cfgpdflayout {OneColumn}      % Modo de página {OneColumn,SinglePage}
\def\cfgpdfmenubar {true}          % Muestra el menú del lector
\def\cfgpdfpageview {FitBV}        % {Fit,FitH,FitV,FitR,FitB,FitBH,FitBV}
\def\cfgpdfsecnumbookmarks {true}  % Número de la sec. en marcadores del pdf
\def\cfgpdftoolbar {true}          % Muestra barra de herramientas lector pdf
\def\cfgshowbookmarkmenu {false}   % Muestra menú marcadores al abrir el pdf
\def\indexdepth {4}                % Profundidad de los marcadores
\def\pdfcompilecompression {9}     % Factor de compresión del pdf (0-9)
\def\pdfcompileobjcompression {2}  % Nivel compresión objetos del pdf (0-3)
\def\usepdfmetadata {true}         % Añade metadatos al pdf compilado

% NOMBRE DE OBJETOS
\def\nameltappendixsection {Anexo} % Etiqueta sección en anexo/apéndices
\def\nameltwfigure {Figura}        % Etiqueta leyenda de las figuras
\def\nameltwsrc {Código}           % Etiqueta leyenda del código fuente
\def\nameltwtable {Tabla}          % Etiqueta leyenda de las tablas
\def\namemathcol {Corolario}       % Nombre de los colorarios
\def\namemathdefn {Definición}     % Nombre de las definiciones
\def\namemathej {Ejemplo}          % Nombre de los ejemplos
\def\namemathlem {Lema}            % Nombre de los lemas
\def\namemathobs {Observación}     % Nombre de las observaciones
\def\namemathprp {Proposición}     % Nombre de las proposiciones
\def\namemaththeorem {Teorema}     % Nombre de los teoremas
\def\namereferences {Referencias}  % Nombre de la sección de referencias

% CONFIGURA BEAMER
% https://latex.ppizarror.com/doc/beameruserguide.pdf
\mode<presentation> {
	
	% CONFIGURACIONES GENERALES
	\def\colnumbers {2}
	\def\colseparation {0.03\paperwidth}
	\def\posterheight {82}
	\def\posterwidth {65}
	
	% CONFIGURA LOS BLOQUES, VALORES EN [EM]
	\def\blockmarginbottom {1}
	\def\blockpaddingbottom {0.6}
	\def\blockpaddingleft {0.6}
	\def\blockpaddingright {0.6}
	\def\blockpaddingtop {0.75}
	
	% ESTILO BLOQUES ALERT/EXAMPLE
	\def\blockaemarginbottom {1}
	\def\blockaepaddingbottom {0.6}
	\def\blockaepaddingleft {0.6}
	\def\blockaepaddingright {0.6}
	\def\blockaepaddingtop {0.75}
	
	% CONFIGURACIÓN DE COLORES Y FUENTES
	% Básicos
	\setbeamercolor{palette primary}{fg=black,bg=white}
	\setbeamercolor{palette secondary}{fg=black,bg=white}
	\setbeamercolor{palette tertiary}{bg=black,fg=white}
	\setbeamercolor{palette quaternary}{fg=black,bg=white}
	\setbeamercolor{structure}{fg=mitred}
	
	% Bibliografía
	\setbeamercolor{bibliography item}{fg=black}
	\setbeamercolor{bibliography entry author}{fg=black}
	\setbeamercolor{bibliography entry title}{fg=black}
	\setbeamercolor{bibliography entry location}{fg=black}
	\setbeamercolor{bibliography entry note}{fg=black}
	\setbeamertemplate{bibliography item}[text]
	
	% Bloque
	\setbeamercolor{block title}{fg=mitred,bg=white}
	\setbeamercolor{block separator}{bg=black}
	\setbeamercolor{block body}{fg=black,bg=white}
	\setbeamerfont{block title}{size=\Large,series=\bfseries}
	
	% Bloque de alerta
	\setbeamercolor{block alerted title}{fg=white,bg=gray}
	\setbeamercolor{block alerted separator}{bg=lgray}
	\setbeamercolor{block alerted body}{fg=white,bg=gray}
	
	% Bloque de ejemplo
	\setbeamercolor{block example title}{fg=mitred,bg=lgray}
	\setbeamercolor{block example separator}{bg=black}
	\setbeamercolor{block example body}{fg=black,bg=lgray}
	
	% Cabecera y Títulos
	\setbeamercolor{headline}{fg=black,bg=lgray}
	\setbeamercolor{heading}{fg=black}
	\setbeamerfont{headline}{}
	\setbeamerfont{headline title}{size=\huge,series=\bfseries}
	\setbeamerfont{headline author}{size=\Large}
	\setbeamerfont{headline institute}{size=\small}
	\setbeamerfont{heading}{series=\bfseries}
	
	% Navegación
	\beamertemplatenavigationsymbolsempty
	
	% Pie de página
	\setbeamerfont{footline}{size=\normalsize}
	
}
